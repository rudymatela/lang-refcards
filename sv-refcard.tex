%% sv-refcard.tex
%
% Copyright 2022 Rudy Matela
%
% This text is available under (at your option):
%   * Creative Commons Attribution-ShareAlike 3.0 Licence
%   * GNU Free Documentation License version 1.3 or Later
%

\documentclass[14pt]{refcard} % 12pt when with more stuff?
\usepackage[T1]{fontenc}
\usepackage[swedish]{babel}
\usepackage[utf8]{inputenc}
\usepackage{multicol}
\usepackage{tipa}
\usepackage{wasysym}
\usepackage[hidelinks]{hyperref}
\geometry{a4paper,landscape,nohead,includefoot,margin=18pt,left=18pt,right=18pt,bottom=32pt,footskip=16pt}
%\renewcommand{\familydefault}{\sfdefault}

\usepackage{selinput}

\title{Swedish Language Reference Card}

\cright{
	\scriptsize
	~~ Copyright 2022, Rudy Matela --- compiled \today{} \\
	~~ \url{https://matela.com.br/sv-refcard.pdf} ~ \url{https://github.com/rudymatela/lang-refcards}
}{
	\scriptsize
	This text is available under
	the Creative Commons Attribution-ShareAlike 3.0 Licence, ~ \\
	\textbf{or} (at your option), the GNU Free Documentation License version 1.3 or Later.
}
\makeatletter
\cfoot{\scriptsize\@title{}\\v0.0}
\makeatother

\newcommand{\singlec}[1]{\multicolumn{1}{c}{#1}}
\newcommand{\doublec}[1]{\multicolumn{2}{c}{#1}}
\newcommand{\doublel}[1]{\multicolumn{2}{l}{#1}}
\newcommand{\doubler}[1]{\multicolumn{2}{r}{#1}}
\newcommand{\triplec}[1]{\multicolumn{3}{c}{#1}}
\newcommand{\quadruplel}[1]{\multicolumn{4}{l}{#1}}
\newcommand{\quadrupler}[1]{\multicolumn{4}{r}{#1}}


\begin{document}\centering

\maketitle
\vspace{-1ex}
Referenskort för Svenskt Språk

\subsubsection{Användbara fraser -- Useful phrases}
\vspace{-1ex}
\begin{tabular}{ll}
ja                     & yes                            \\
nej                    & no                             \\[1ex]

Hej!                   & Hi!                            \\
Hej då!                & Bye!                           \\
God morgon.            & Good morning.                  \\
God dag.               & Good day. \emph{(formal)}      \\
God kväll.             & Good evening. \emph{(formal)}  \\[1ex]

tack                   & thanks                         \\
tack                   & please                         \\
snälla                 & please                         \\
tack tack              & thanks \emph{(informal)}       \\
varsågod               & You're welcome.                \\[1ex]

ursäkta                & I'm sorry. / Excuse me.        \\[1ex]

välkommen              & Welcome.                       \\[1ex]
\end{tabular}

\pagebreak

\subsubsection{Nummer, Siffra -- Numbers}
\vspace{-1ex}
\begin{tabular}{rl rl rl}
0 & noll    &  10 & tio              &   20 & tjugo            \\
1 & en, ett &  11 & elva             &   30 & trettio          \\
2 & två     &  12 & tolv             &   40 & fyrtio           \\
3 & tre     &  13 & tretton          &   50 & femtio           \\
4 & fyra    &  14 & fjorton          &   60 & sextio           \\
5 & fem     &  15 & femton           &   70 & sjuttio          \\
6 & sex     &  16 & sexton           &   80 & åttio            \\
7 & sju     &  17 & sjutton          &   90 & nittio           \\
8 & åtta    &  18 & arton            &  100 & hundra           \\
9 & nio     &  19 & nitton           & 1000 & tusen            \\
\end{tabular}


\end{document}
