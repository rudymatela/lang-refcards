%% sv-refcard.tex
%
% Copyright 2022 Rudy Matela
%
% This text is available under (at your option):
%   * Creative Commons Attribution-ShareAlike 3.0 Licence
%   * GNU Free Documentation License version 1.3 or Later
%

\documentclass[14pt]{refcard} % 12pt when with more stuff?
\usepackage[T1]{fontenc}
\usepackage[swedish]{babel}
\usepackage[utf8]{inputenc}
\usepackage{multicol}
\usepackage{tipa}
\usepackage{wasysym}
\usepackage[hidelinks]{hyperref}
\geometry{a4paper,landscape,nohead,includefoot,margin=18pt,left=18pt,right=18pt,bottom=32pt,footskip=16pt}
%\renewcommand{\familydefault}{\sfdefault}

\usepackage{selinput}

\title{Swedish Language Reference Card}

\cright{
	\scriptsize
	~~ Copyright 2022, Rudy Matela --- compiled \today{} \\
	~~ \url{https://matela.com.br/sv-refcard.pdf} ~ \url{https://github.com/rudymatela/lang-refcards}
}{
	\scriptsize
	This text is available under
	the Creative Commons Attribution-ShareAlike 3.0 Licence, ~ \\
	\textbf{or} (at your option), the GNU Free Documentation License version 1.3 or Later.
}
\makeatletter
\cfoot{\scriptsize\@title{}\\v0.2}
\makeatother

\newcommand{\singlec}[1]{\multicolumn{1}{c}{#1}}
\newcommand{\doublec}[1]{\multicolumn{2}{c}{#1}}
\newcommand{\doublel}[1]{\multicolumn{2}{l}{#1}}
\newcommand{\doubler}[1]{\multicolumn{2}{r}{#1}}
\newcommand{\triplec}[1]{\multicolumn{3}{c}{#1}}
\newcommand{\quadruplel}[1]{\multicolumn{4}{l}{#1}}
\newcommand{\quadrupler}[1]{\multicolumn{4}{r}{#1}}


\begin{document}\centering

\maketitle
\vspace{-1ex}
Referenskort för Svenskt Språk

\subsubsection{Användbara fraser -- Useful phrases}
\vspace{-1ex}
\begin{tabular}{ll}
ja                     & yes                            \\
nej                    & no                             \\
nja                    & nyes (maybe)                   \\ [1ex]

Hej!                   & Hi!                            \\
Hej då!                & Bye!                           \\
God morgon.            & Good morning.                  \\
God dag.               & Good day. \emph{(formal)}      \\
God kväll.             & Good evening. \emph{(formal)}  \\[1ex]

tack                   & thanks                         \\
tack                   & please                         \\
snälla                 & please                         \\
tack tack              & thanks \emph{(informal)}       \\
Tack så mycket!        & Thanks so much!                \\
varsågod               & You're welcome.                \\
ursäkta                & Excuse me.                     \\
förlåt                 & I'm sorry.                     \\
välkommen              & Welcome.                       \\[1ex]

Jag vet inte.          & I don't know.                  \\
Jag förstår inte.      & I don't understand.            \\
%Jag förstår inte Svenska. & I don't understand Swedish. \\[1ex]
Jag heter \emph{namn}. & I'm called \emph{name}.        \\
Precis.                & Exactly.                       \\[1ex]

Skål!                  & Cheers! (Toast)                \\
\end{tabular}

\pagebreak

\subsubsection{Nummer, Siffra -- Numbers}
\vspace{-1ex}
\begin{tabular}{rl rl rl}
0 & noll    &  10 & tio             &  100 & hundra           \\
1 & en, ett &  11 & elva            & 1000 & tusen            \\
2 & två     &  12 & tolv            &   20 & tjugo            \\
3 & tre     &  13 & tretton         &   30 & trettio          \\
4 & fyra    &  14 & fjorton         &   40 & fyrtio           \\
5 & fem     &  15 & femton          &   50 & femtio           \\
6 & sex     &  16 & sexton          &   60 & sextio           \\
7 & sju     &  17 & sjutton         &   70 & sjuttio          \\
8 & åtta    &  18 & arton           &   80 & åttio            \\
9 & nio     &  19 & nitton          &   90 & nittio           \\
\end{tabular}

\subsubsection{Ordinal Numbers}
\vspace{-.5ex}
\begin{tabular}{@{}rl rl rl}
     &         & last & sist    &  0:e & nollte \\
 1:a & första  &  2:a & andra   &  3:a & tredje \\
 4:a & fjärde  &  5:a & femte   &  6:a & sjätte \\
 7:a & sjunde  &  8:a & åttonde &  9:a & nionde \\
10:e & tionde  & 11:e & elfte   & 12:e & tolfte \\
\end{tabular} \\
\begin{tabular}{@{}rl rl}
 13:e & trettonde   &   30:e & trettionde \\
 14:e & fjortonde   &   40:e & fyrtionde  \\
 15:e & femtonde    &   50:e & femtionde  \\
 16:e & sextonde    &   60:e & sextionde  \\
 17:e & sjuttonde   &   70:e & sjuttionde \\
 18:e & artonde     &   80:e & åttionde   \\
 19:e & nittonde    &   90:e & nittionde  \\
 20:e & tjugonde    &  100:e & hundrade  \\
 21:a & tjugoförsta & 1000:e & tusende   \\
 22:a & tjugoandra  & {\footnotesize 1 000 000:e}   & miljonte \\
 23:e & tjugotredje & {\scriptsize 1 000 000 000:e} & miljarte \\
\end{tabular}


\pagebreak

\subsubsection{Mat och dryck -- Food and bevrg.}

\begin{tabular}{@{} r@{\ \ }l @{\hspace{-1ex}} rl @{}}
%1234567890 & 01234567890 & 01234567890 & 01234567890 \\
vatten      & water       & dryck       & beverage    \\
öl          & beer        & vin         & wine        \\
kaffe       & coffee      & te          & tea         \\
socker      & sugar       & mataffär    & supermarket \\
bröd        & bread       & smör        & butter      \\
juice, jos  & juice       & is          & ice         \\
kött        & beef, meat  & fläsk       & pork        \\
kyckling    & chicken     & ägg         & egg         \\
anka        & duck        & fisk        & fish        \\
ost         & cheese      & skinka      & ham         \\
tonfisk     & tuna        & lax         & salmon      \\
hummer      & lobster     & räka        & shrimprawn  \\
blötdjur    & mollusk     & kräftdjur   & crustacean  \\
allergi     & allergy     & medicin     & medicine    \\
mandarin    & tangerine   & banan       & banana      \\
\end{tabular}


\subsubsection{Questions -- Frågor}

\begin{tabular}{rl rl}

Vad?       & What?     & Vem?       & Who?      \\
När?       & When?     & Vilken?    & Which?    \\
Var?       & Where?    & Vart?      & Where to? \\
Varför?    & Why?      & Hur?       & How?      \\
\end{tabular}


\subsubsection{Verbs}

\begin{tabular}{rl rl}
vara  & to be         & är     & is, are \\
tala  & to speak      & talar  & speaks, talks \\
prata & to speak      & pratar & speaks, talks \\
säga  & to say        & säger  & says, tells \\
ska   & \multicolumn{3}{l}{will, shall} \\
\end{tabular}


\subsubsection{Misc}

\begin{tabular}{rl}
än    & yet  \\
båder & both \\
här   & here \\
hit   & here \\
nära  & near \\
\end{tabular}


\subsubsection{Transportation}

\begin{tabular}{@{} r@{\ \ }l @{\hspace{-2ex}} r@{\ \ }l @{}}
väg        & road       & bro        & bridge \\
cykel      & bicycle    & bil        & car    \\
motorcykel & motorcycle & lastbil    & truck  \\
buss       & bus        & tåg        & train  \\
flygplats  & airport    & tunnelbana & metro  \\
flygplan   & airplane   & båt        & boat   \\
           &            & färja      & ferry  \\
\end{tabular}



\end{document}
