%% pl-refcard.tex
%
% Copyright 2019 Rudy Matela
%
% This text is available under (at your option):
%   * Creative Commons Attribution-ShareAlike 3.0 Licence
%   * GNU Free Documentation License version 1.3 or Later
%

\documentclass[12pt]{refcard}
\usepackage[T1]{fontenc}
\usepackage[polish]{babel}
\usepackage[utf8]{inputenc}
\usepackage{multicol}
\usepackage{tipa}
\usepackage{wasysym}
\usepackage[hidelinks]{hyperref}
\geometry{a4paper,landscape,nohead,includefoot,margin=18pt,left=18pt,right=18pt,bottom=32pt,footskip=16pt}
%\renewcommand{\familydefault}{\sfdefault}

\usepackage{selinput}

\title{Polish Language Reference Card}

\cright{
	\scriptsize
	~~ Copyright/Prawo autorskie 2019, Rudy Matela --- skompilowany \today{} \\
	~~ \url{https://matela.com.br/pl-refcard.pdf} ~ \url{https://github.com/rudymatela/lang-refcards}
}{
	\scriptsize
	This text is available under
	the Creative Commons Attribution-ShareAlike 3.0 Licence, ~ \\
	\textbf{or} (at your option), the GNU Free Documentation License version 1.3 or Later.
}
\makeatletter
\cfoot{\scriptsize\@title{}\\v0.4}
\makeatother

\newcommand{\singlec}[1]{\multicolumn{1}{c}{#1}}
\newcommand{\doublec}[1]{\multicolumn{2}{c}{#1}}
\newcommand{\doublel}[1]{\multicolumn{2}{l}{#1}}
\newcommand{\doubler}[1]{\multicolumn{2}{r}{#1}}
\newcommand{\triplec}[1]{\multicolumn{3}{c}{#1}}
\newcommand{\quadruplel}[1]{\multicolumn{4}{l}{#1}}
\newcommand{\quadrupler}[1]{\multicolumn{4}{r}{#1}}


\begin{document}\centering

\maketitle
\vspace{-1ex}
Karta Referencyjna Języka Polskiego

\subsubsection{Przydatne zwroty - Useful phrases}
\vspace{-1ex}
\begin{tabular}{ll}
\singlec{tak -- yes}   & \singlec{Dobrze! -- Ok!}       \\
\singlec{nie -- no}    & \singlec{tak sobie -- so-so}   \\[1ex]

Cześć!                 & Hi! Bye! \emph{(Informal.)}    \\
Dzień dobry.           & Good day.                      \\
Dobry wieczór.         & Good evening.                  \\
Dobranoc.              & Good night.                    \\
Do zobaczenia.         & Goodbye.  See you.             \\[1ex]

proszę                 & please                         \\
Dzięki!                & Thanks! \emph{(Informal.)}     \\
Dziękuję.              & Thank you.                     \\
Dziękuję bardzo.       & Thank you very much.           \\[1ex]

Nie ma za co.          & You're welcome.                \\
Proszę bardzo.         & You're welcome.                \\
Nie ma sprawy.         & You're welcome.                \\
Nie ma problemu.       & You're welcome.                \\[1ex]

Przepraszam.           & I'm sorry. / Excuse me.        \\[1ex]

Witaj.                 & Welcome.                       \\
Miło mi.               & Pleased to meet you.           \\[1ex]

Poproszę \emph{rzecz}. & A/Some \emph{thing} please.    \\
Chciałbym \emph{rzecz}. & I'd like a/some \emph{thing}. \\
Gdzie jest \emph{rzecz}? & Where is \emph{thing/place}? \\
Co to jest?            & What is this?                  \\
To jest \emph{rzecz}.  & This is a/the \emph{thing}.    \\[1ex]

Na zdrowie!            & Cheers! \emph{To health!}      \\
Sto lat!               & Cheers! \emph{100 years!}      \\[1ex]

Pomocy! / Ratunku!     & Help!                          \\
\end{tabular}


\subsubsection{Numery, Liczby -- Numbers}
\vspace{-1ex}
\begin{tabular}{r@{~~}l@{\hspace{1em}}r@{~~}l@{\hspace{1em}}r@{~~}l@{\hspace{1em}}r@{~~}l@{\hspace{1em}}r@{~~}l}
  $+$ & plus  & $1$ & jeden  & $4$ & cztery & $7$ & siedem   \\
  $-$ & minus & $2$ & dwa    & $5$ & pięć   & $8$ & osiem    \\
  $0$ & zero  & $3$ & trzy   & $6$ & sześć  & $9$ & dziewięć \\
\end{tabular}\\[1ex]
\begin{tabular}{rl@{\hspace{2em}}rl}
10 & dziesięć         &   20 & dwadzieścia      \\
11 & jedenaście       &   30 & trzydzieści      \\
12 & dwanaście        &   40 & czterdzieści     \\
13 & trzynaście       &   50 & pięćdziesiąt     \\
14 & czternaście      &   60 & szesdziesiąt     \\
15 & piętnaście       &   70 & siedemdziesiąt   \\
16 & szesnaście       &   80 & osiemdziesiąt    \\
17 & siedemnaście     &   90 & dziewięćdziesiąt \\
18 & osiemnaście      &  100 & sto              \\
19 & dziewiętnaście   & 1000 & tysiąc           \\
\end{tabular}

\vspace{-1ex}
\subsubsection{Ordinals}
\vspace{-1ex}
\begin{tabular}{r@{~}l@{\hspace{2em}}r@{~}l@{\hspace{2em}}r@{~}l}
1. & pierwszy  & 4. & czwarty  & 7. & siódmy    \\
2. & drugi     & 5. & piąty    & 8. & ósmy      \\
3. & trzeci    & 6. & szósty   & 9. & dziewiąty \\
\end{tabular}\\[1ex]
\begin{tabular}{r@{~}l@{\hspace{5em}}r@{~}l}
10. & dziesiąty   &  16. & szesnasty     \\
11. & jedenasty   &  17. & siedemnasty   \\
12. & dwanasty    &  18. & osiemnasty    \\
13. & trzynasty   &  19. & dziewiętnasty \\
14. & czternasty  &  20. & dwudziesty    \\
15. & piętnasty   & last & ostatni       \\
\end{tabular}

\vspace{-1ex}
\subsubsection{Pytania -- Questions}
\vspace{-1ex}
\begin{tabular}{r@{ -- }lr@{ -- }l}
co?       & what?       &
kto?      & who?        \\
kiedy?    & when?       &
gdzie?    & where?      \\
dokąd?    & where to?   &
skąd?     & where from? \\
dlaczego? & why?        &
jak?      & how?        \\
czy?      & yes or no?  &
ile?      & how much?   \\
\quadruplel{Co to znaczy \emph{słowo}? -- What does \emph{word} mean?} \\
\quadruplel{Jak się mówi po polsku \emph{word}?}  \\[-.5ex]
\quadrupler{-- How do we say \emph{word} in polish?} \\
\quadruplel{Czy mówisz po angielsku? \scriptsize-- Do you speak english?} \\
\end{tabular}


\subsubsection{Polski Alfabet -- The Polish Alphabet}
\vspace{-1ex}
\begin{tabular}{r@{\,-- }c@{ --\,}l@{\hspace{1.2em}}r@{\,-- }c@{ --\,}l@{\hspace{0em}}r@{\,-- }c@{ --\,}l}
\textipa{[a]}   & \bf a & a   & \textipa{[j]}           & \bf j & jot                    & \textipa{[s]}           & \bf s & es    \\
\textipa{[\~O]} & \bf ą & ą   & \textipa{[k]}           & \bf k & ka                     & \textipa{[C]}           & \bf ś & eś    \\
\textipa{[b]}   & \bf b & be  & \textipa{[l]}           & \bf l & el                     & \textipa{[t]}           & \bf t & te    \\
\textipa{[ts]}  & \bf c & ce  & \textipa{[w]}           & \bf ł & eł                     & \textipa{[u]}           & \bf u & u     \\
\textipa{[tC]}  & \bf ć & cie & \textipa{[m]}           & \bf m & em                     & \multicolumn{1}{c@{}}{} & \bf v & fał   \\
\textipa{[d]}   & \bf d & de  & \textipa{[n]}           & \bf n & en                     & \textipa{[v]}           & \bf w & wu    \\
\textipa{[E]}   & \bf e & e   & \textipa{[\textltailn]} & \bf ń & eń                     & \textipa{[ks]}          & \bf x & iks   \\
\textipa{[\~E]} & \bf ę & ę   & \textipa{[O]}           & \bf o & o                      & \textipa{[1]}           & \bf y & igrek \\
\textipa{[f]}   & \bf f & ef  & \textipa{[u]}           & \bf ó & \scriptsize o z kreską & \textipa{[z]}           & \bf z & zet   \\
\textipa{[g]}   & \bf g & gie & \textipa{[p]}           & \bf p & pe                     & \textipa{[\textctz]}    & \bf ź & ziet  \\
\textipa{[x]}   & \bf h & ha  & \multicolumn{1}{c@{}}{} & \bf q & ku                     & \textipa{[Z]}           & \bf ż & żet   \\
\textipa{[i]}   & \bf i & i   & \textipa{[r]}           & \bf r & er                     \\
\end{tabular} \\[1ex]
\begin{tabular}{r@{\,--\,}lr@{\,--\,}lr@{\,--\,}lr@{\,--\,}l}
\bf ch  & \textipa{[x]}               &
\bf sz  & \textipa{[S]}               &
\bf ci  & \textipa{[tCi]}             &
\bf zi  & \textipa{[\textctz{}i]}     \\
\bf cz  & \textipa{[tS]}              &
\bf rz  & \textipa{[Z]}/\textipa{[S]} &
\bf si  & \textipa{[Ci]}              &
\bf dzi & \textipa{[d\textctz{}i]}    \\
\end{tabular}

\vspace{-1ex}
\paragraph{Similar IPA sounds}
\vspace{-1ex}
\begin{tabular}{r@{ $\approx$ }l@{\hspace{2em}}r@{ $\approx$ }l@{\hspace{2em}}r@{ $\approx$ }l@{\hspace{2em}}r@{ $\approx$ }l}
\textipa{[C]} & \textipa{[S]} & \textipa{[\textctz]} & \textipa{[Z]} &
\textipa{[x]} & \textipa{[h]} & \textipa{[r]} & \textipa{[\*r]} \\
\end{tabular}

\vspace{-1ex}
\subsubsection{Pronouns and possesive pronouns}
\vspace{-1ex}
\begin{tabular}{rlrl}
--  & --        & swój, swoja, swoje & \footnotesize \emph{subject's} \\
ja  & I         & mój, moja, moje    & my \\
ty  & you       & twój, twoja, twoje & yours \\
on  & he        & jego & his  \\
ona & she       & jej  & her \\
ono & it        & jego & its  \\
my  & we        & nasz, nasza, nasze & ours \\
wy  & you       & wasz, wasza, wasze & yours \\
oni, one & they & ich & theirs \\
\end{tabular}

\vspace{-1ex}
\subsubsection{Partykuły -- Particles}
\vspace{-1ex}
\begin{tabular}{@{}r@{\,--\,}l@{ }r@{\,--\,}l@{ }r@{\,--\,}ll}
i         & and         &
z         & from/with   &
od        & from        &
\\

lub       & or          &
bez       & without     &
do        & to/until    &
\\

ani       & nor         &
w         & in/at       &
gdy       & while       &
\\

dla       & for         &
na        & onto/atop   &
nad       & over        &
\\

gdyby     & if          &
nigdy     & never       &
zawsze    & always      &
\\
\end{tabular}



\subsubsection{Koniugacja -- Conjugation}

\subsubsection{być -- to be}
\vspace{-1ex}
\begin{tabular}{rlll}
\scriptsize            & \footnotesize present
\scriptsize            & \footnotesize future
\scriptsize            & \footnotesize imperative \\[-1ex]
\scriptsize            & \footnotesize teraźniejszy
\scriptsize            & \footnotesize przyszły
\scriptsize            & \footnotesize rozkazujący \\
\scriptsize ja         & jestem    & będę      & niech będę   \\
\scriptsize ty         & jesteś    & będziesz  & bądź         \\
\scriptsize on,ona,ono & jest      & będzie    & niech będzie \\
\scriptsize my         & jesteśmy  & będziemy  & bądźmy       \\
\scriptsize wy         & jesteście & będziecie & bądżcie      \\
\scriptsize oni, one   & są        & będą      & niech będą   \\[1ex]
\end{tabular}

\noindent
\begin{tabular}{rr@{ }lr@{ }l}
\scriptsize       & \doublec{\footnotesize past}
\scriptsize       & \doublec{\footnotesize conditional} \\[-1ex]
\scriptsize       & \doublec{\footnotesize przeszły}
\scriptsize       & \doublec{\footnotesize przypuszczający} \\[-1ex]
\scriptsize       & \male~ & ~~\female & \male~ & ~~\female \\
\scriptsize ja    & byłem & byłam & byłbym & byłabym \\
\scriptsize ty    & byłeś & byłaś & byłbyś & byłabyś \\
\scriptsize on... & \doublec{był było była} & \doublec{byłby byłoby byłaby} \\
\scriptsize my    & byliśmy & byłyśmy & bylibyśmy & byłybyśmy \\
\scriptsize wy    & byliście & byłyście & bylibyście & byłybyście \\
\scriptsize oni/e & byli & były & byliby & byłyby \\
\end{tabular}

\subsubsection{(z)robić -- to do}
\vspace{-1ex}
\begin{tabular}{rlll}
\scriptsize          & \tiny teraźniejszy
\scriptsize          & \footnotesize przyszły
\scriptsize          & \footnotesize rozkazujący \\
\scriptsize ja       & robię   & będę      robić & --          \\
\scriptsize ty       & robisz  & będziesz  robić & rób         \\
\scriptsize on...    & robi    & będzie    robić & niech robi  \\
\scriptsize my       & robimy  & będziemy  robić & róbmy       \\
\scriptsize wy       & robicie & będziecie robić & róbcie      \\
\scriptsize oni, one & robią   & będą      robić & niech robią \\[1ex]
\end{tabular}

\noindent
\begin{tabular}{r@{ }lr@{ }l}
\doublec{\footnotesize przeszły} &
\doublec{\footnotesize przypuszczający} \\[-1ex]
\male~ & ~~\female & \male~ & ~~\female \\
robiłem    & robiłam    & robiłbym     & robiłabym    \\
robiłeś    & robiłaś    & robiłbyś 	& robiłabyś 	  \\
\doublec{robił robiła robiło} & \doublec{robiłby robiłaby robiłoby} \\
robiliśmy  & robiłyśmy  & robilibyśmy  & robiłybyśmy  \\
robiliście & robiłyście & robilibyście & robiłybyście \\
robili     & robiły     & robiliby     & robiłyby     \\
\end{tabular}

\vspace{-1ex}
\subsubsection{Inne czasowniki -- Other verbs}
\vspace{-1ex}
\begin{tabular}{lllll}
& \footnotesize to have
& \footnotesize can/may
& \footnotesize to know
& \footnotesize to must \\
                   & mieć    & móc     & znać    & musieć  \\[.5ex]
\scriptsize ja     & mam     & mogę    & znam    & muszę   \\
\scriptsize ty     & masz    & możesz  & znasz   & musisz  \\
\scriptsize on/a/o & ma      & może    & zna     & musi    \\
\scriptsize my     & mamy    & możemy  & znamy   & musimy  \\
\scriptsize wy     & macie   & możecie & znacie  & musicie \\
\scriptsize oni/e  & mają    & mogą    & znają   & muszą   \\
\end{tabular} \\[1ex]
\begin{tabular}{r@{ -- }l@{\hspace{-1ex}}r@{ -- }l}
chcieć   & to want & dać        & to give \\
iść      & to go   & chodzić    & to go \\
mówić    & to say  & powiedzieć & to say  \\
wiedzieć & to know & lubić      & to like \\
rozumieć & to understand & szukać & to search \\
\end{tabular}

\vspace{-1ex}
\subsubsection{Czas -- Time}
\vspace{-1ex}
\begin{tabular}{@{}r@{ -- }l@{\hspace{-2ex}}r@{ -- }l}
zegar        & clock       &
kalendarz    & calendar    \\
godzina      & hour        &
dzień        & day         \\
minuta       & minute      &
tydzień      & week        \\
sekunda      & second      &
miesiąc      & month       \\
chwila       & moment      &
rok, lat     & year, years \\
teraz        & now         &
poniedziałek & monday      \\
teraźniejszy & present     &
wtorek       & tuesday     \\
przeszły     & past        &
środa        & wednesday   \\
przyszły     & future      &
czwartek     & thursday    \\
dzisiaj      & today       &
piątek       & friday      \\
jutro        & tomorrow    &
sobota       & saturday    \\
wczoraj      & yesterday   &
niedziela    & sunday      \\
\end{tabular}

\vspace{-1ex}
\subsubsection{Kraje i Języka -- Countries and Languages}
\vspace{-1ex}
\begin{tabular}{r@{ -- }lr@{ -- }l}
%kraj       & country   & language    & język       \\[1ex]
Polska     & Poland    & polish      & polski      \\
Anglia     & England   & english     & angielski   \\
%Belgia     & Belgium   & --          & --          \\
%Białoruś   & Belarus   & belarussian & białoruski  \\
Brazylia   & Brazil    & portuguese  & portugalski \\
%Czechy     & Czech R.  & czech       & czeski      \\
%Francja    & France    & french      & francuski   \\
%Hiszpania  & Spain     & spanish     & hiszpański  \\
%Litwa      & Lithuania & lithuanian  & litewski    \\
%Niemcy     & Germany   & german      & niemiecki   \\
%Portugalia & Portugal  & portuguese  & portugalski \\
%Rosja      & Russia    & russian     & rosyjski    \\
%Słowacja   & Slovakia  & slowak      & słowacki    \\
Szwecja    & Sweden    & swedish     & szwedzki    \\
%Ukraina    & Ukraine   & ukranian    & ukraiński   \\
Włochy     & Italy     & italian     & włoski      \\
\doubler{Unia Europejska} & \doublel{European Union} \\
%\doubler{Rzeczpospolita Polska} & \doublel{Republic of Poland} \\
%\multicolumn{2}{r}{Wielka Brytania}       & \multicolumn{2}{l}{Great Britain} \\
%\multicolumn{2}{r}{Zjednoczone Królestwo} & \multicolumn{2}{l}{United Kingdom} \\
%\multicolumn{2}{r}{Stany Zjednoczone}     & \multicolumn{2}{l}{United States} \\
\end{tabular}

\vspace{-1ex}
\subsubsection{Kierunki i Transport -- Directions and Tr.}
\vspace{-1ex}
\begin{tabular}{@{\hspace{-1ex}}r@{\,--\,}l@{\hspace{1.25ex}}r@{\,--\,}l@{\hspace{-0.75ex}}r@{\,--\,}l}
lewo     & left & prawo & right & góra & top    \\
tutaj/tu & here & tam   & there & dół  & bottom \\
zachód & west & wschód & east & północ & north \\
blisko            & near      & daleko            & far       & południe & south \\
\end{tabular} \\
%\begin{tabular}{r@{ -- }l}
%północny   wschód & northeast \\
%południowy zachód & southwest \\
%północny   zachód & northwest \\
%południowy wschód & southeast \\
%\end{tabular}\\
\begin{tabular}{r@{ -- }l@{\hspace{1ex}}r@{ -- }l}
droga & road & ulica & street \\
pociąg & train & bilety & ticket \\
taksówka & taxi  &  autobus & bus \\
auto  & car & samochód & car \\
%rower & bicycle & samolot & aircraft \\
sklep & shop & paszport & passport \\
%stacja & station & lotnisko & airport \\
%motocykl   & motocykle & motorcycle   \\
\end{tabular}

\vspace{-1ex}
\subsubsection{Jedzenie i Napoje -- Food and Beverages}
\vspace{-1ex}
\begin{tabular}{@{}r@{ -- }l@{\hspace{-1em}}r@{ -- }l}
woda & water & mleko & milk    \\
piwo & beer  & wino & wine     \\
kawa & coffee & cukier & sugar \\
chleb       & bread    & masło       & butter    \\
%sok         & juice     & lód         & ice       \\
wołowina    & beef       & wieprzowina & pork      \\
kurczak     & chicken    & ryba        & fish      \\
homar       & lobster    & krewetka    & shrimprawn \\
mięczak     & mollusk    & skorupiak   & crustacean \\
alergia     & allergy    & lekarstwo   & medicine \\
pączek      & doughnut   & pączki      & doughnuts \\
pierogi     & dumplings  & gołąpki     & cabbage roll \\
bigos       & \footnotesize hunter's stew   & banan       & banana    \\
rachunek & bill & rezerwacja & reservation \\
\end{tabular}
%\begin{tabular}{r@{ -- }lr@{ -- }l}
%pomarańcza  & orange    \\
%limonka     & lime      \\
%cytryna     & lemon     \\
%winogrono   & grape     \\
%mandarynka  & tangerine \\
%klementynka & tangerine \\
%\end{tabular}

\vspace{-1.33ex}
\subsubsection{Wewnątrz -- Indoors}
\vspace{-1ex}
\begin{tabular}{r@{ -- }lr@{ -- }l}
łazienka & bathroom & toaleta & bathroom \\
sala & room & korytarz & corridor \\
drzwi & door & okno & window \\
stół     & table & krzesło  & chair      \\
\end{tabular}

\vspace{-1.33ex}
\subsubsection{Kolory -- Colors}
\vspace{-1ex}
\begin{tabular}{r@{ -- }lr@{ -- }l}
czerwony & red &
biały & white \\
zielony & green &
czarny & black \\
niebieski & blue &
szary & grey \\
żółty & yellow &
pomarańczowy & orange \\
\end{tabular}


\end{document}
